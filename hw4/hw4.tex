\documentclass[a4paper, 12pt]{article}

\usepackage{amsmath}
\usepackage{amssymb}
\usepackage{fancyhdr}
\usepackage[margin = 1in]{geometry}
\usepackage{listings}
\usepackage{enumitem}

\usepackage[margin = 1in]{geometry}
\usepackage{color}

\lstset{frame=tb,
	language=Java,
	aboveskip=3mm,
	belowskip=3mm,
	showstringspaces=false,
	columns=flexible,
	basicstyle={\small\ttfamily},
	numbers=none,
	numberstyle=\tiny\color{gray},
	keywordstyle=\color{blue},
	commentstyle=\color{dkgreen},
	stringstyle=\color{mauve},
	breaklines=true,
	breakatwhitespace=true,
	tabsize=3
}
\definecolor{dkgreen}{rgb}{0,0.6,0}
\definecolor{gray}{rgb}{0.5,0.5,0.5}
\definecolor{mauve}{rgb}{0.58,0,0.82}

\newcommand{\question}[1] {\vspace{.25in} \hrule\vspace{0.5em}
	\noindent{\bf #1} \vspace{0.5em}
	\hrule \vspace{.10in}}
\renewcommand{\part}[1] {\vspace{10pt} {\bf (#1)}}

\setlength{\parindent}{0pt}
\setlength{\parskip}{5pt plus 1pt}

\pagestyle{fancyplain}
\DeclareMathOperator{\maxx}{max}

\begin{document}
	\medskip
	
	\thispagestyle{plain}
	\begin{center}
		{\Large ICCS 313: Assignment 4} \\
		Tawan Chaeyklinthes u5980963 \\
		Date : 12 Nov 2018
	\end{center}

\question{Problem 1}

Assuming that we have $P_0,P_1,...,P_k $ which indicates prices for the rod. Let $value(x)$ be the maximum value of a rod of length $x$. We can then write the recurrence as follows:

\begin{equation*}
	value(x)=\begin{cases}
			0, & \text{if $x\leqslant 0$.}\\	
			\underset{i \leqslant k }{max}(P_i + value(x-i) -cut(i,x)), & \text{if $x > 0$.}
		\end{cases}	
\end{equation*}

Let $cut(i,x)$ be an indicator to whether or not are we cutting the rod. If we are cutting the rod, return $c$ ,otherwise, return 0.
\begin{equation*}	
	cut(i,x) = \begin{cases}
	0 , & \text{if $i==x$.}\\
	c, & \text{otherwise.}
	\end{cases}
\end{equation*}

This can be translated into code as such:
\begin{lstlisting}
	def rod_cutting(n,P):
		values = [0]*(n+1)
		for x in range(1,n+1):
			for i in range(len(P)):
				val = values[x-i] + P[i]
				if(i != x):
					val -= c
				values[x] = max(val,values[x])
		return values[n] 
		
\end{lstlisting}

\question{Problem 2}

Let array $best\_seq[x]$ keep the best contiguous subsequence from elements index $0,1,...,x$.

The algorithm is to keep track of the running sum while updating the $best\_seq$ .If the running sum is $< 0$, for instance at index $i$ , reset the sum.

This can be written in code as follows:

\begin{lstlisting}
	def max_sequence(S):
		best_seq = [-INF]
		current_seq = []
		sum = 0
		for i in range(len(S)):
			sum += S[i]
			current_seq.append(S[i]) 
			if(sum > sum(best_seq)):
				best_seq = current_seq.copy()	
			if(sum < 0):
				sum = 0
				current_seq = []	
		return best_seq
\end{lstlisting}

\question{Problem 3}

Let $longest\_p(i,j)$ be the function that gets the longest palindrome of range $i \text{ to } j$. We can write a recurrence relation as follows:

\begin{equation*}
	longest\_p(i,j) = \begin{cases}
		\text{empty string}, & \text{if } i>j.\\
		s[i], & \text{if } i == j.\\
		longest\begin{cases}
			longest\_p(i+1,j)\\
			longest\_p(i,j-1)
		\end{cases} + \underset{\text{if } s[i] == s[j]}{longest\_p(i+1,j-1)} , & \text{ otherwise.}
	\end{cases}
\end{equation*}

This can be translated into codes as follow:

\begin{lstlisting}
	def longest_p(i,j):
		if(i == j):
			return s[i]
		if(i > j):
			return ""
		if(memo[i][j] != null):
			return memo[i][j]
		else:
			longest_palin_without_j = longest(i,j-1)
			longest_palin_without_i = longest(i+1,j)
			longest_palin = longest(longest_palin_without_j,
									longest_palin_without_i)
			if(s[i] == s[j]):
				longest_palin = s[i] +longest(i+1,j-1) + s[j]
			memo[i][j] = longest_palin
			return longest_palin
\end{lstlisting}

There are $O(n^2)$ sub-problems and each costing $O(1)$. Therefore the total running time is $O(n^2)$.




\end{document}