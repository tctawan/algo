\documentclass[a4paper, 12pt]{article}

\usepackage{amsmath}
\usepackage{amssymb}
\usepackage{fancyhdr}
\usepackage{listings}
\usepackage{enumitem}

\usepackage[margin = 1in]{geometry}
\usepackage{color}

\lstset{frame=tb,
	language=Java,
	aboveskip=3mm,
	belowskip=3mm,
	showstringspaces=false,
	columns=flexible,
	basicstyle={\small\ttfamily},
	numbers=none,
	numberstyle=\tiny\color{gray},
	keywordstyle=\color{blue},
	commentstyle=\color{dkgreen},
	stringstyle=\color{mauve},
	breaklines=true,
	breakatwhitespace=true,
	tabsize=3
}
\definecolor{dkgreen}{rgb}{0,0.6,0}
\definecolor{gray}{rgb}{0.5,0.5,0.5}
\definecolor{mauve}{rgb}{0.58,0,0.82}

\newcommand{\question}[1] {\vspace{.25in} \hrule\vspace{0.5em}
	\noindent{\bf #1} \vspace{0.5em}
	\hrule \vspace{.10in}}
\renewcommand{\part}[1] {\vspace{.10in} {\bf (#1)}}

\setlength{\parindent}{0pt}
\setlength{\parskip}{5pt plus 1pt}

\pagestyle{fancyplain}

\begin{document}
	\medskip
	
	\thispagestyle{plain}
	\begin{center}
		{\Large ICCS 313: Assignment 3}\\
		Tawan Chaeyklinthes u5980963\\
		Date: 07 Oct 2018
	\end{center}

\question{Problem 1}
A tree is an acyclic and connected undirected graph which infer that it must have $n-1$ edges.
\begin{lstlisting}
	def isTree(V,E):
		n = |V|, m = |E|
		if(m != n-1):
			return False
		else:
			Run bfs on the graph.
			Keep track of all visited nodes in a set, S.
			if len(S) == n:
				return True
			else:
			 return False
\end{lstlisting}
Therefor, the running time of this algorithm is  $= O(m+n)$.
\question{Problem 2}
\underline{Lemma :} If $G$ is a DAG, then $G$ has a node with no entering edges.
 
\underline{Proof :} 

For the sake of contradiction, suppose that $G$ is a DAG and every node has at least one incoming edges. Pick a node $v$ and follow an edge backward to $u$. Repeat this step for node $u$ and follow an edge backwards to some node $x$. Keep doing this until we meet the same node twice, say $w$. Now, we have a sequence  $w \rightarrow w$  which is a cycle. Thus, graph $G$ is not a DAG .Hence we have our contradiction. Therefore, a DAG has at least one node with no incoming edges.

\underline{Lemma :} If $G$ is a DAG, then $G$ has a node with no outgoing edges.

\underline{Proof :}

For the sake of contradiction, suppose that $G$ is a DAG and every node has at least one outgoing edges. Pick a node $v$ and follow an edge forwards to $u$. Repeat this step for node $u$ and follow an edge forwards to some node $x$. Keep doing this until we meet the same node twice, say $w$. Now, we have a sequence  $w \rightarrow w$  which is a cycle. Thus, graph $G$ is not a DAG .Hence we have our contradiction. Therefore, a DAG has at least one node with no outgoing edges.  

Using the two lemmas, we can conclude that if $G$ is a DAG, $G$ has a node with no outgoing edges and a node with no incoming edges.

\question{Problem 3}
\underline{Obs :} For some node $v$, if $deg(v) = n/2 $, $v$  must be connected to $n/2+1$ nodes.

\underline{Lemma :} Let $G$ be a graph of $n$ nodes, where $n$ is an even number. If every node of $G$ has degree at least $n/2$, then $G$is connected.

\underline{Proof: }

For the sake of contradiction, let assume that $G$ is a disconnected graph comprise of graphs $G_1$ and $G_2$ and every node has degree at least $n/2$. In $G_1$, since every node has to has a degree at least $n/2$, $|V_1| \geqslant n/2+1$ and therefore $|V_2| = |V|-|V_1|$ which is $\leqslant n/2-1$. With $\leqslant n/2-1$ nodes in $G_2$, the maximum degree of the nodes in $G_2$ is $(n/2-1)-1 = n/2-2$. Hence we have a contradiction. Therefore if every node has degree at least $n/2$, the graph has to be connected.

\question{Problem 4}
We can find the order in which to serve the customer by sorting their service time, $t_i$ so that $t_1\leqslant t_2 \leqslant t_3 ... \leqslant t_n $ whereby customer with $t_1$ will be served first and customer with $t_n$ will be served last.

\question{Problem 5}
\part{a}
Let's look at this problem as a graph where each node represent a person and each edge shows the friendship between any 2 persons.
\begin{description}[font=$\bullet$]
	\item[] Firstly, looping through the list edges ,we find out the degree of each node.
	\item[] Keep the degree and name of each person as a pair.
	\item[] Sort these pairs according to their degrees in ascending order.
	\item[] Iterate if the first pair has degree $< 5$, we removed it and update its neighbors.
	\item[] Sort these pairs.
	\item [] Repeat the previous 2 steps until first pair have degree $\geqslant 5$.
	\item [] Now invite all remaining pairs.  
\end{description}
Let $S$ be the set that is produced by greedy and $T$ be the set of optimal solution. And let the set that is removed from $S$ and $T$ be $q_1,q_2,...,q_n$ and $r_1,r_2,...,r_m$ respectively, where $q_n$ and $r_m$ are just some pairs.

\underline{Lemma 1 :} At the end of each iteration degree of $q_1,q_2,...,q_k < 5$.

\underline{Proof :}

Initialized: The removed set is empty. Thus the claim is trivially true. 

Maintenance: Assuming that the in iteration $k$ , the claim is true. In iteration $k+1$ , if the degree of first element in the remaining set is $\geqslant 5$ then the loop will terminate. However , if degree $> 5$, then it will be removed from $S$. Thus the $q_k$ will be add to removed set and now degree of $q_1,q_2,...,q_{k-1},q_k < 5$. 

Terminate: The loop will terminate when the degree first pair element $\geqslant 5$. Thus, $q_1,q_2,...,q_n < 5$.


\underline{Lemma 2 :}
The greedy will returns an optimal set $S$ . 

\underline{Proof :}

For the sake of contradiction, let assume that $S$ is not optimal which means that the number of pairs removed from $S$ is more than that of $T$. In other word lets assume that $m = k$ , the $n = k+1$. From the lemma above, we know that $deg(i_{k+1}) < 5$. Thus it can be said that in $T$ there are still pairs with degree $< 5$ left. Thus we have a contradiction since $T$ is supposed to be an optimal solution.

\part{b} Let $E$ be the list of edges:
\begin{lstlisting}
	def invite(E):
		Make adjTable ,map name to neighbours.
		Make degTable ,map name to degree.
		Make removedSet ,store removed pairs.
		Make PriorityQueue<Pair>, pq , higher priority for lower degree pairs.
		for name in degTable.key():
			p=(name,degTable.get(name))
			pq.add(p)
		while (!pq.isEmpty && pq.peek < 5):
			name, deg = pq.poll()
			removedSet.add(name)
			for neighbour in adjTable.get(name):
				if !(removedSet.contains(neighbour)): 
					Update degree of neighbour in degTable.
					Create pair (neighbour,degTable.get(neightbour))
					Add new pair to pq.
		invited = []
		for name in adjTable.key():
			if(!(removed.contains(name))):
				invited.append(name)
		return invited
\end{lstlisting}
\underline{Runtime:}

The first for-loop take $O(n\log n)$ to make first priority queue. In the while-loop, the maximum number of poll $pq$ has to make is $n$ times. This is because of the if-condition in the for-loop which make sure that a removed node is never added back to $pq$. Therefore, polling will take a total of $O(n\log n)$. In the second for-loop, the maximum number of iteration is $\sum deg(neighbour) = O(m)$ and $pq.add()$ takes $O(\log n)$. Thus, the cost of this for-loop is $O(m\log n)$.Lastly, the last for-loop will cost $O(n)$. Thus, the total running time is : $O(n\log n) + O(n\log n) + O(m\log n ) = 2O(n\log n) + O(m \log n) = O((n+m)\log n)$.
\end{document}
