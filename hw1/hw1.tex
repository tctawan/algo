\documentclass[a4paper, 12pt]{article}

\usepackage{amsmath}
\usepackage{amssymb}
\usepackage{fancyhdr}
\usepackage[margin = 1in]{geometry}

\newcommand{\question}[1] {\vspace{.25in} \hrule\vspace{0.5em}
\noindent{\bf #1} \vspace{0.5em}
\hrule \vspace{.10in}}
\renewcommand{\part}[1] {\vspace{10pt} {\bf (#1)}}

\setlength{\parindent}{0pt}
\setlength{\parskip}{5pt plus 1pt}

\pagestyle{fancyplain}

\begin{document}
	\medskip
	
	\thispagestyle{plain}
	\begin{center}
		{\Large ICCS 313: Assignment 1} \\
		Tawan Chaeyklinthes u5980963 \\
		Date : 14 Sep 2018
	\end{center}
\question{Problem 1}
\part {a}
$3n^3 + 75n^2 + 8\log_2n \in \mathcal{O}(n^3)$ \\

Proof:
\begin{align*}
3n^3 + 75n^2 + 8\log_2n & \leqslant 3n^3 + 75n^3 + 8^3
&&\text{; for }n \geqslant 1 \\
& \leqslant 90n^3
\end{align*}
Therefor this is true when $c = 5 \text{ and } n_0 = 1$.\\

\part{b} $1+3+5+...+(2n-1) \in \mathcal{O}(n^2)$\\

Proof: We know that $1+2+3+...+(2n-1) = \sum_{i=1} ^{n} (2i-1)$. Which can be written as: 
\begin{align*}
	\sum_{i=1}^{n} 2i - \sum_{i=1}^{n} 1 & = 2({\frac{n(n+1)}2}) - n \\
	&= n^2 + n -2n
\end{align*} 
Thus,
\begin{align*}
	n^2 \leqslant n^2 &&\text{ ; for } n\geqslant 1 
\end{align*}
So,this is true when $c=1 \text{ and } n_0 = 1$.

\part{c}
$1+2+4+8+...+2n^2 \in \mathcal{O}(2^n)$\\

Proof: Let $S(n) = 1+2+4+8+...+2n^2$,
\begin{align*}
	S(n) &= 2^0+2^1+2^2+...+2^n\\
	2S(n) &= 2^1 + 2^2 +...+2^n+2^{n+1}\\
	&= S(n) - 1 + 2^{n+1}\\
	S(n)  &= 2^{n+1} -1\\
	&= 2(2^n) -1
\end{align*}
Thus,
\begin{align*}
	 2(2^{n}) -1 \leqslant 2(2^n) && \text{ ; for } n\geqslant 1
\end{align*}
So, this is true when $c = 2 \text{ and } n_0 = 1$.

\part{d}
$1 + \frac12 + \frac{1}{4} + ...+\frac{1}{2^n} \in \mathcal{O}(1)$\\

Proof: Let $S(n) = 1 + \frac12 + \frac{1}{4} + ...+\frac{1}{2^n}$
\begin{align*}
	S(n) &= \frac{1}{2^0} + \frac{1}{2^1} + \frac{1}{2^2} + ... +\frac{1}{2^n}\\
	\frac{S(n)}{2} &= \frac{1}{2^1} + \frac{1}{2^2} + \frac{1}{2^3} +... +\frac{1}{2^n} +\frac{1}{2^{n+1}} \\
	&= S(n) -1 +\frac{1}{2^{n+1}}\\
	-S(n)(\frac{1}{2}) &= \frac{1}{2^{n+1}} -1	\\
	S(n) &= 2 - \frac{1}{2^n}
\end{align*}
Therefor, we can see that:
\begin{align*}
	2 - \frac{1}{2^n} \leqslant 2(1) && \text{ ; for } n \geqslant 1
\end{align*}
So,this is true when $c=2 \text{ and } n_0 = 1$.
\question{Problem 2}

\part{a} We need to show that all the elements in A remains in A after the loop.

\part{b} Loop-invariant: At the start of each iteration, in the sub-array $A[j...n], A[j]$ is the smallest.

Initialized: At the start, $j = n$ so the sub-array is $A[n]$. Since $A[n]$ is the only element, it is the smallest.

Maintenance: Let $k = \text{ iteration number}$. Suppose that $A[j_k] < A[{j-1}_k] $, the value of $A[j_k]$ and $A[{j-1}_k]$ will then be swapped. The value of $A[{j-1}_k]$ is now the smallest since $A[j_k]$ was originally the smallest in $A[j_k...n]$.Therefore, in the next iteration, value of $A[j_{k+1}]$ will be the smallest in $A[j_{k+1}...n]$. If $A[j_k] > A[{j-1}_k] $, it means that $A[{j-1}_k]$ is now the smallest in $A[{j-1}_k...]$. There will be no swapping so in the next iteration, $A[j_{k+1}]$ will be the smallest in $A[j_{k+1}...n]$. Thus, the invariant is maintained.

Termination: The for-loop will terminate at $j = i$. From the loop-invariant, this will give us a sub-array $A[i...n]$ where $A[i]$ is the smallest.

\part{c} Loop-invariant: At the start of each iteration, the sub-array $A[1...i]$ is sorted

Initialized: At the start i = 1, the sub-array A[1] is already sorted.

Maintenance: Let $i = k$ where $k$ is an iteration of the for-loop. Using invariant from part (b), we know that $A[k]$ is the smallest in $A[k...n]$. And with invariant from part (c), we know that $A[1...k-1]$ is already sorted. We also know that as the value of $i$ increases, the length of final sub-array in inner-loop decreases. Therefore, we are sure that $A[k-1]$ must be the smallest value in $A[k-1,k,...n]$ the iteration before. Thus, at the end of the iteration, we would have $A[1...k]$ which is sorted.

Termination: The loop will terminate when $i = n$. With our invariant, this will give us a sub-array $A[1...n]$ that is sorted. By observation, the sub-array is our array. Thus, the algorithm gives us a sorted array upon termination.

\part{d} The worst-case is when the array is sorted in descending order. The running-time would be :
\begin{align*}
	T(n) &= \sum_{i=2}^{n-1} i\\
	&\approx \frac{n(n+1)}{2}\\
	&= \mathcal{O}(n^2)
\end{align*}
Which is the same as the worst-case running time for insertion sort.
\question{Problem 3}
$S(n) = 1^c + 2^c +3^c +...+n^c$

\part{a} $S(n) \text{ is } \mathcal{O}(n^c+1)$

Proof:
\begin{align*}
	1^c + 2^c +3^c +...+n^c \leqslant n(n^c) && \text{ ; for } n\geqslant 1
\end{align*}
So, this is true when constant $= n$ and $n_0 = 1$.

\part{b}

Proof:
\begin{align*}
	1^c + 2^c +3^c +...+n^c &\geqslant (\frac{n}{2})^c + ... + n^c && \text{ ; for } n\geqslant 1 \\
	&\geqslant (\frac{n}{2})(\frac{n}{2})^c \\
	& \geqslant (\frac{n}{2})^{c+1}\\
	& \geqslant (\frac{1}{2})^{c+1} n^{c+1}
\end{align*}
So, this is true when constant $= (\frac{1}{2})^{c+1}$ and $n \geqslant 1$.


\question{Problem 4}
\part{a} Observations: $2 = 2$ bits, $4 = 3$ bits, $8 = 4$ bits, $2^n = n+1$ bits.

So the big-O is: $\mathcal{O}(\log_2n + 1) =\mathcal{O}(\log_2n)$.

\part{b} The number $n$ is being divided by 2 for each loop. So the numbers of times that $n$ has to be divided by 2 before reaching 1 is $\log_2n$ and the total iteration is $\log_2n +1$. Thus the big-O is $\mathcal{O}(\log_2n)$.

\part{c} Each iteration that the code makes, the input size is reduced by half. So we can write the running time as follow:
\begin{align*}
	T(n) &= n^3 + (\frac{n}{2})^3 +(\frac{n}{2^2})^3 + (\frac{n}{2^3})^3 +... + 1\\
	&= n^3 + (\frac{n^3}{8}) + (\frac{n^3}{8^2})+(\frac{n^3}{8^3})+ ... + 1\\
	\frac{T(n)}{8} &= (\frac{n^3}{8}) +(\frac{n^3}{8^2})+(\frac{n^3}{8^3})+(\frac{n^3}{8^4}) +... + 1 + \frac{1}{8}\\
	&= T(n) -n^3 + \frac18\\
	\frac78T(n) &= n^3 - \frac18\\
	T(n) &= \frac{8n^3}{7} - 7 \\
	&= \mathcal{O}(n^3)
\end{align*}
So the running time is $\mathcal{O}(n^3)$.
\question{Problem 5}
\part{a} Function foo() takes in a sorted list, integers $m$ and $val$. The function first search for an interval of size $m+1$ where the value $val$ could be inside. If such interval is not found, function returns -1. Else, it would then search the interval one by one for the value $val$. If the value is found, function will return the index of the value, otherwise ,it returns -1.

\part{b} \underline{Lemma 1} : The first while-loop will terminate with the right interval (if exist) for $m \geqslant 1$.

\underline{Proof} : When $m \geqslant 1$, first while-loop will terminate on 2 conditions. Firstly, the loop will break if an interval where $alist[left] \leqslant val$ and $alist[right] \geqslant val$ since there is a statement checking for this condition. Hence, this break will give us the correct interval and will terminate. If such interval doesn't exist, the value of $left$ will keep increasing until $left \geqslant length$ or $alist[left] > val$. This will then return -1 on line 10 which indicates that $val$ doesn't exist in the array.

\underline{Lemma 2} : The second while-loop will return the  index of $val$ or -1.

\underline{Proof} : In second while-loop, it will iterate over the range that we got from the first while-loop one element at a time. If the element is equal to $val$ it will return that index, thus, returning the index of $val$. However, if $val$ is not in the array, i will keep increasing till $ i > r$ or $alist[i] > val$ which will terminate the while-loop and returns -1.

With this 2 lemmas, it can be seen that this code will return the right answer. 

\part{c} The worst-case is when $m = 1$ and the value to find is more than last index. This will give us a running time of: 
\begin{align*}
	T(n) = n = \mathcal{O}(n)
\end{align*}

\part{d} The maximum number of comparison made in first-loop is $\frac{n}{m}$ and the second loop is $m$. So to find the minimum we take the derivative of function: $\frac{n}{m} + m $.
\begin{align*}
	\frac{d}{dm}(\frac{n}{m} + m ) &= \frac{-n}{m^2} + 1 
\end{align*}
To find the minimum, we let : $\frac{d}{dm}(\frac{n}{m} + m ) = 0$:
\begin{align*}
	\frac{-n}{m^2} + 1 &= 0 \\
	\frac{n}{m^2} &= 1\\
	n &= m^2\\
	m &= \sqrt{n}
\end{align*}
Therefore, the $m = \sqrt{n}$ for the minimum comparison. 

\question{Problem 6}

\part{a} $\mathcal{O}(n^3)$

\part{b} $\mathcal{O}(nlogn)$

\part{c} $\mathcal{O}(3^n)$

\part{d} $\mathcal{O}(n^3)$

\part{e} $\mathcal{O}(n^4)$

\end{document}